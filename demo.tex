\section{Demo details}
\label{sec:demo}

%Overall theme of the demo and what are goals are: Showing how PBS-metrics can be
%integrated into the DB admin's interface. Also showing why consistency metrics
%are important and how PBS can be used to measure this etc.

We will present an end-to-end demonstration that will show how metrics defined
by PBS can be used to capture the impact of consistency on user experience. We
will also highlight how database administrators can monitor and modify system
parameters to trade-off consistency for latency.\\[-.5em]

The demonstration will consist of several components:

%Specific demo setup details: What is the app going to be -- What tables is it
%going to contain and how is this stored in Cassandra ?

\textbf{Web Application:} We will implement a Twitter-like web
application such as Twissandra. We plan to make a web-interface that
will be available to all SIGMOD attendees and a mobile application
that can be used to post messages about the conference.

\textbf{Data Sets:} In addition to the messages posted by SIGMOD
attendees, we plan to replay a corpus of 4,937,001 Tweets from
conversations obtained from the Twitter firehose between February and
July 2011~\cite{ritter2010unsupervised}.  This will help us explore
some of the issues that arise with a heavy workload.

\textbf{Distributed Datastore:} We will host a 50-node Cassandra
cluster on EC2 to support our web application. The Cassandra instances
will use a table to store the Tweets published and a separate table to
store the follower/following relationship information.\\[-.5em]

%Demo screens detail - We will have two screens and what each one will show. How
%can the audience interact with the demo ?

To illustrate the utility of consistency metrics, we will present two
interactive, ``behind-the-scenes'' DBA- and application-oriented
dashboards:

\textbf{DB admin interface:} We will show a monitoring console that
measures and plots the consistency and latency over 10-second time
intervals.  Additionally the admin interface will allow users to
modify system parameters like the read repair chance and set
consistency SLAs for read and write
operations. Figure~\ref{fig:pbs-demo-screenshot} shows an example DBA
dashboard will look.

\textbf{User interface:} The user interface screen will show our
web-application described previously and will allow attendees to post
and read messages. We will also provide an optional setting that
allows users to see how old messages are, to manually inspect the
end-to-end delay from the write to their chosen interface. As messages
are submitted to the Twitter clone, attendees can also witness
user-visible latency and consistency in the monitoring interface.\\[-.5em]

Consistency is particularly interesting under changing environmental
conditions, so we will allow participants to control parameters like
system latency and the performance of several replicas:

\textbf{Chaos console:} We will provide a separate interface that can
be used to inject message delays, model messages being dropped, and
artificially slow Cassandra instances to demonstrate their effect on
consistency. In addition to improving engagement, this will allow
attendees to both induce consistency anomalies and understand the
impact of several common failure modes.


